\documentclass[11pt]{article}
\usepackage[a4paper,left=1.5cm,right=1.5cm,top=1.5cm,bottom=1.5cm]{geometry}
\usepackage{fancyhdr}
\renewcommand{\headrulewidth}{1pt}
\fancyhead[C]{\textbf{[LINGI2132] Languages \& Translators}}
\fancyhead[L]{Assignment 2}
\fancyhead[R]{Group 55}

\usepackage[T1]{fontenc}
\usepackage[utf8]{inputenc}
\usepackage[english]{babel}
\usepackage[nounderscore]{syntax}
\usepackage{listings}
\let\syntleft\relax
\let\syntright\relax
\setlength{\grammarindent}{3.5em}
\usepackage{graphicx}
\usepackage{subcaption}
\usepackage{mathtools,amssymb, mathrsfs,amsthm}
\usepackage[binary-units=true,separate-uncertainty = true,multi-part-units=single]{siunitx}
\usepackage{float}
\setlength{\parskip}{1ex plus 0.5ex minus 0.2ex}
\newcommand{\hsp}{\hspace{20pt}}
\newcommand{\HRule}{\rule{\linewidth}{0.5mm}}
\graphicspath{{img/}}
\usepackage{caption}
\usepackage{textcomp}
\usepackage{array}
\usepackage{color}
\usepackage{tabularx,booktabs}
\usepackage{titlesec}
\usepackage{wrapfig}
\titlespacing{\section}{0pt}{\parskip}{-\parskip}
\titlespacing{\subsection}{0pt}{\parskip}{-\parskip}
\titlespacing{\subsubsection}{0pt}{\parskip}{-\parskip}
\pagestyle{fancy}
\usepackage{minted}
\newcommand{\java}[1]{\mintinline{java}{#1}}
\lstdefinestyle{myjvm}{basicstyle=\ttfamily,
	language=JVMIS,
	morekeywords={if_icmple},
	keywordstyle=\color{codepurple},
	stringstyle=\color{codegreen}
}
\newcommand{\jvm}[1]{\lstinline[style=myjvm]|#1|}
\usepackage{csquotes}
\usepackage[linktoc=all]{hyperref}
\hypersetup{breaklinks=true}
\usepackage{tikz}
\usetikzlibrary{automata,positioning}
\DeclareMathOperator{\first}{first}
\DeclareMathOperator{\follow}{follow}
\newcommand{\id}{\mathrm{id}}
\newcommand{\E}{\textnormal{E}}
\newcommand{\Ep}{\textnormal{E'}}
\newcommand{\T}{\textnormal{T}}
\newcommand{\Tp}{\textnormal{T'}}
\newcommand{\F}{\textnormal{F}}
\renewcommand{\epsilon}{\varepsilon}
\renewcommand{\theta}{\vartheta}
\renewcommand{\kappa}{\varkappa}
\renewcommand{\rho}{\varrho}
\renewcommand{\phi}{\varphi}
\usepackage{import}
\newcommand{\orc}{\mathbin{\mathrm{or}}}
\newcommand{\andc}{\mathbin{\mathrm{and}}}
\newcommand{\notc}{\mathop{\mathrm{not}}}
\definecolor{codegreen}{rgb}{0.25,0.5,0.35}    % comments
\definecolor{codepurple}{rgb}{0.5,0,0.35}      % keywords

\begin{document}
\import{../rec_descent/}{LL1_grammar.tex}

\section{Programming Directly in Java Bytecode}
Using \java{CLEmitter}, it is quite simple to create a ``hand-written'' class file.
In order to specify the output directory for the generated file, one can use the \java{destinationDir} method.
A \java{CLEmitter} uses an \java{ArrayList} to store the various instructions that are needed.
Similarly to how this was done in the provided examples of \java{CLEmitter} usage on the lecture slides and in the \texttt{tests/clemitter} folder, the class is created using the \java{addClass} method, and an implicit no-arg constructor is added.
For the \java{gcd} method, \java{ClassToGenerate} mainly uses \jvm{iload_0}, \jvm{iload_1} (to load the first and second arguments, resp.), \jvm{istore_0}, \jvm{istore_1} (to (over)write the first and second arguments, resp.), as well as the \jvm{isub} command to subtract two numbers.
These are all called using \java{CLEmitters}'s \java{addNoArgInstruction} method.

In order to implement the various control structures (the \java{while} loop and \java{if} block), \java{CLEmitter} has an \java{addBranchInstruction} method, which takes a first argument with the branch instruction to execute (\jvm{goto} or \jvm{if_icmple}), and a second argument with the name of the label to which one needs to branch.
Labels can be specified using \java{CLEmitter}'s \java{addLabel} instruction, which takes a \java{String} as argument.

To test the correctness of the \java{Generator} class, one can use the Fernflower (or any other) decompiler, which translates a compiled class file into a regular java file.
This output can then be compared with the original java file.

\section{Lexical Analysis}
\subsection{Hand-written Compiler}
In the \java{Scanner}, skipping over block comments is done by first detecting the start of a block comment in the \java{getNextToken} method.
Once this has been detected, one enters a \java{while} loop which exits as soon as the end of the block comment has been detected.

\subsection{JavaCC Compiler}
Using JavaCC and lexical states, a neat solution is the following:

\noindent\texttt{
SKIP: \{"/*": IN\_BLOCK\_COMMENT\}\\
< IN\_BLOCK\_COMMENT > SKIP: \{< \(\sim\)[] >\}\\
< IN\_BLOCK\_COMMENT > SKIP: \{"*/": DEFAULT\}}

It works as follows: when the start of a block comment is matched in the  \texttt{DEFAULT} state, it is skipped and the state is switched to \texttt{IN\_BLOCK\_COMMENT}.
Any other character seen in this new state is skipped, unless the JavaCC sees the end of a block comment, in which case this is skipped and the state is switched back to \texttt{DEFAULT}.

\subsection{Tests}
In order to test whether this works correctly, multi-line comments to a test file, and we verified everything was still compiling correctly.

\section{Parsing and Semantic Analysis}
\end{document}